\documentclass[a4paper,12pt]{article}
\title{Assignment 2}
\author{ZHU Cai\\
Student No.: 09902510R\\
Department of Applied mathematics}
\usepackage{graphics}
\usepackage{amsmath}
\usepackage{Sweave}
\begin{document}
\maketitle
This assignment is programed and calculated using R-2.10.1.\\
\section*{Question 1}
\noindent (a) From the definition, the rate of return, r, can be calculate as $r = \frac{X_1-X_0}{X_0}$:
\begin{gather}
X_0 = 1e6 + 0.5\times u\\
X_1 = 0.6\times4e6 + 0.4\times u = 2.4e6 + 0.4\times u
\end{gather}
Thus, the rate of return
\begin{equation}
r = \frac{1.4e6 - 0.1u}{1e6 + 0.5u}
\end{equation}

\noindent (b) There is a simple answer: we can make the rate of return 4e6 in probability of
1, so we should buy 4e6 pieces of insurance. If it does not rain, the investor will get 4e6,
and if it does rain, the investor will also get 4e6 from those insurance. Thus, there is
no risk. Now, lets do some calculation.\\
The variance of return can be definited as follows:\\
\begin{equation}
var = 0.6\times(4e6-2.4e6-0.4u)^2 + 0.4\times(u-2.4e6-0.4u)^2
\end{equation}
\begin{Schunk}
\begin{Sinput}
> variance <- function(x) {
+     fun <- 0.6 * (4 - 2.4 - 0.4 * x)^2 + 0.4 * (x - 2.4 - 0.4 * 
+         x)^2
+     return(fun)
+ }
> result <- optimize(variance, interval = c(3, 6))
> result
\end{Sinput}
\begin{Soutput}
$minimum
[1] 4

$objective
[1] 0
\end{Soutput}
\end{Schunk}
From the result, we can see we should buy 4000000 units of insurance.
\section*{Question 3}
\noindent For (a)
\begin{Schunk}
\begin{Sinput}
> rm(list = ls(all = TRUE))
> a <- c(1, 1, 0, -1, 1, 2, 1, -1, 0, 1, 2, -1, 1, 1, 1, 0)
> A <- matrix(a, nrow = 4, ncol = 4, byrow = TRUE)
> B <- c(0, 0, 0, 1)
> A
\end{Sinput}
\begin{Soutput}
     [,1] [,2] [,3] [,4]
[1,]    1    1    0   -1
[2,]    1    2    1   -1
[3,]    0    1    2   -1
[4,]    1    1    1    0
\end{Soutput}
\begin{Sinput}
> solve(A, B)
\end{Sinput}
\begin{Soutput}
[1]  1.0 -0.5  0.5  0.5
\end{Soutput}
\end{Schunk}
\noindent For (b)
\begin{Schunk}
\begin{Sinput}
> rm(list = ls(all = TRUE))
> d <- c(1, 1, 0, -0.4, -1, 1, 2, 1, -0.8, -1, 0, 1, 2, -0.8, -1, 
+     0.4, 0.8, 0.8, 0, 0, 1, 1, 1, 0, 0)
> D <- matrix(d, nrow = 5, ncol = 5, byrow = TRUE)
> E <- c(0, 0, 0, 0.6, 1)
> D
\end{Sinput}
\begin{Soutput}
     [,1] [,2] [,3] [,4] [,5]
[1,]  1.0  1.0  0.0 -0.4   -1
[2,]  1.0  2.0  1.0 -0.8   -1
[3,]  0.0  1.0  2.0 -0.8   -1
[4,]  0.4  0.8  0.8  0.0    0
[5,]  1.0  1.0  1.0  0.0    0
\end{Soutput}
\begin{Sinput}
> solve(D, E)
\end{Sinput}
\begin{Soutput}
[1]  5.000000e-01 -2.220446e-16  5.000000e-01  1.250000e+00  2.775558e-16
\end{Soutput}
\end{Schunk}
\noindent For (c)
\begin{Schunk}
\begin{Sinput}
> rm(list = ls(all = TRUE))
> cov <- matrix(c(1, 1, 0, 1, 2, 1, 0, 1, 2), nrow = 3, ncol = 3, 
+     byrow = TRUE)
> Return <- c(0.4, 0.8, 0.8)
> omega1 <- solve(cov, Return)
> omega1/sum(omega1)
\end{Sinput}
\begin{Soutput}
[1] 0.5 0.0 0.5
\end{Soutput}
\begin{Sinput}
> omega2 <- solve(cov, (Return - 0.1))
> omega2/sum(omega2)
\end{Sinput}
\begin{Soutput}
[1] 0.3333333 0.1666667 0.5000000
\end{Soutput}
\end{Schunk}
Thus, (a)$w_1 = 1, w_2 = -0.5,w_3 =0.5, \mu=0.5$;\\
\noindent (b) $w_1=0.5, w_2 =0, w_3=0.5, \lambda=1.25, \mu=0$\\
\noindent (c) when $\lambda=1, \mu=0$, we have $w_1 =0.5, w_2 =0, w_3 =0.5$
and when $r_f =0.1$, we have $w_1 = 0.33, w_2 = 0.17, w_3 = 0.5$

\section*{Question 4}
\begin{Schunk}
\begin{Sinput}
> rm(list = ls(all = TRUE))
> data <- read.csv("C:/Documents and Settings/zc/Desktop/mywork/course-IS/assign 2/assig2q4.csv")
> stocks <- data[, 2:16]
> Ocal <- function(data) {
+     T <- 22
+     n <- length(data)
+     K <- n - T
+     O <- rep(0, length = K)
+     for (j in 1:K) {
+         O[j] <- (data[(j + T)] - data[j])/data[j]
+     }
+     return(O)
+ }
> returns <- apply(stocks, 2, Ocal)
> dim(returns)
\end{Sinput}
\begin{Soutput}
[1] 106  15
\end{Soutput}
\begin{Sinput}
> r <- apply(returns, 2, mean)
> COV <- cov(returns)
> N <- ncol(COV)
> N
\end{Sinput}
\begin{Soutput}
[1] 15
\end{Soutput}
\begin{Sinput}
> vector <- rep(-1, length = N)
> A1 <- cbind(COV, vector)
> dim(A1)
\end{Sinput}
\begin{Soutput}
[1] 15 16
\end{Soutput}
\begin{Sinput}
> vector <- c(rep(1, length = N), 0)
> A2 <- rbind(A1, vector)
> dim(A2)
\end{Sinput}
\begin{Soutput}
[1] 16 16
\end{Soutput}
\begin{Sinput}
> B1 <- c(rep(0, length = N), 1)
> omega1 <- solve(A2, B1)
> round(omega1, 4)
\end{Sinput}
\begin{Soutput}
 [1] -0.3703 -0.0470 -0.0546  0.3686  0.3723  0.4399 -0.0149  0.1366  0.4735
[10] -0.2853  0.0319 -0.0566  0.0249  0.0080 -0.0269  0.0001
\end{Soutput}
\begin{Sinput}
> e <- rep(1, length = N) * (-1)
> r <- -1 * r
> C1 <- cbind(COV, r, e)
> r <- c(r, 0, 0)
> e <- c(e, 0, 0)
> C2 <- rbind(C1, r, e)
> dim(C2)
\end{Sinput}
\begin{Soutput}
[1] 17 17
\end{Soutput}
\begin{Sinput}
> B2 <- c(rep(0, length = N), 0.008, 1)
> omega2 <- solve(C2, B2)
> round(omega2, 4)
\end{Sinput}
\begin{Soutput}
 [1]  0.2404  0.2038  0.1021 -0.2564 -0.2264 -0.5614 -0.1026 -0.1339 -0.2909
[10]  0.1355 -0.0636  0.0645 -0.0197 -0.1193  0.0277 -0.0013 -0.0001
\end{Soutput}
\end{Schunk}
Thus, omega1 is for question (a), and omega2 is for question (b).

\section*{Question 7}
\noindent (a) The Captical market line is as follows:\\
\begin{equation}
r_i = r_f + \frac{\sigma_i}{\sigma_M}(r_M - r_f)
\end{equation}
that is,
\begin{equation}
r_i = 0.03 + \frac{0.17}{0.32}\sigma
\end{equation}
\noindent (b) We know $r_i = 0.3$, from (a), we can calculate $\sigma = 0.51$\\\
\noindent suppose $100\alpha$ percent of money is invested into the risk free asset, we
have the follwoing equation:
\begin{equation}
30\times X_0 = 3\times\alpha X_0 + 20\times(1-\alpha)X_0
\end{equation}
where $X_0 = 1000$, so $\alpha = -0.59$, which means we should borrow $\alpha\times1000 =
590$ at risk free rate, and invest these borrowed money into the risk asset.
\noindent (c) The return rate $r = 0.5\times3 + 0.5\times20 = 11.5$, so we can expect to
have $1000\times(1 + r) = 1115$ at the end of year.
\section*{Question 8}
\noindent From Question 7, we could know that there is a linear relationship between $r_i$
and $\beta_i$. The equation is as follows:
\begin{equation}
r = r_f + (r_M -r_f)\times\beta
\end{equation}
where $r_f$ is the risk free return rate, and $r_M$ is the rate of return on the market potrfolio.
\begin{Schunk}
\begin{Sinput}
> return <- c(0.07, 0.13)
> beta <- c(0.6, 1.5)
> summary(lm(return ~ beta))
\end{Sinput}
\begin{Soutput}
Call:
lm(formula = return ~ beta)

Residuals:
ALL 2 residuals are 0: no residual degrees of freedom!

Coefficients:
            Estimate Std. Error t value Pr(>|t|)
(Intercept)  0.03000         NA      NA       NA
beta         0.06667         NA      NA       NA

Residual standard error: NaN on 0 degrees of freedom
Multiple R-squared:     1,	Adjusted R-squared:   NaN 
F-statistic:   NaN on 1 and 0 DF,  p-value: NA 
\end{Soutput}
\end{Schunk}
\noindent from the result, we have the security market line:\\
\begin{equation}
r = 0.03 + 0.067\times\beta
\end{equation}
Thus, when $\beta$ is 2, r is 0.16.
\end{document}
